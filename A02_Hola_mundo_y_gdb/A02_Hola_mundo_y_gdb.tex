% === A02 - Hola mundo y GDB ===
% David Alejandro Gonzalez Marquez
% dmarquez@dc.uba.ar / fokerman@gmail.com
% https://github.com/fokerman/Orga2Course

\documentclass[aspectratio=169]{beamer}
% \documentclass[handout]{beamer}

% % % Packages
\usepackage[sfdefault]{AlegreyaSans}
\usepackage{inconsolata}
\usepackage{multicol}
\usepackage{multirow}
\usepackage[spanish]{babel}
\usepackage[utf8]{inputenc}
\usepackage{enumerate}
\usepackage{color}
\usepackage{xcolor}
\usepackage[absolute,overlay]{textpos}
  \setlength{\TPHorizModule}{1mm}
  \setlength{\TPVertModule}{1mm}
\usepackage{framed}
\usepackage{mfirstuc} % para poner en mayusculas la primer letra
\usepackage{xspace} % para crear espacios en comandos 
\usepackage{pbox}
\usepackage{tikz}
\usepackage{mathabx}

% % % Beamer config
\usetheme{Pittsburgh}
\usecolortheme[rgb={1,0.48,0.0}]{structure}
\setbeamercolor{block title}{fg=white,bg=verdeuca}
\xdefinecolor{verdeuca}{rgb}{0.0,0.48,0.54}
\xdefinecolor{naranjauca}{rgb}{1,0.48,0.0}
\setbeamercolor{palette quaternary}{fg=white,bg=verdeuca}
\setbeamertemplate{title page}[default][colsep=-4bp, rounded=true] % remove title shadow
\setbeamertemplate{frametitle}[default][colsep=-2bp, shadow=false] % remove frame title shadow
\setbeamertemplate{navigation symbols}{} % remove navigation symbols
\beamertemplatenavigationsymbolsempty

% % % Colors
\definecolor{AzulClaro}{rgb}{.31,.506,.741}
\definecolor{Gris}{gray}{0.8}
\definecolor{Celeste}{rgb}{.255,.41,.884}
\definecolor{Rojo}{rgb}{1, 0, 0}

% % % Rename
\newcommand{\tab}[0]{\hspace{15pt}}

% % % Blocks
\setbeamercolor{block body}{fg=black, bg=black!10}
\setbeamercolor{block title}{fg=black, bg=black!20}
\setbeamercolor{coloredboxstuffNaranja}{fg=naranjauca,bg=black!10} %% PARA LOS BOX
\setbeamercolor{coloredboxstuffVerde}{fg=verdeuca,bg=black!10} %% PARA LOS BOX

% % % Start

\title{\Huge Hola Mundo en ASM}
\author{David Alejandro González Márquez}
\institute{Departamento de Computación\\
Facultad de Ciencias Exactas y Naturales\\
Universidad de Buenos Aires}
\date{}

\begin{document}

\begin{frame}[plain]
    \titlepage 
\end{frame}

\begin{frame}[t]{Hola Mundo ...}
    \begin{block}{Ejercicio}
    Escriba un programa en lenguaje ensamblador que imprima por pantalla:\\
    \begin{center}
    \Large
    \texttt{Hola Mundo}
    \end{center}
    \vspace{.2cm}
    \end{block}
    \vspace{0.7cm}
    \pause
    \begin{center}
    \Huge ¿Cómo?
    \end{center}
\end{frame}

\begin{frame}{Secciones, etiquetas y símbolos}
    Un programa en general se separa en secciones\\
    \pause
    \begin{itemize}
    \item[-] \normalsize \texttt{.data}: Donde declarar variables globales inicializadas. \small (\texttt{DB}, \texttt{DW}, \texttt{DD} y \texttt{DQ}).
    \pause
    \item[-] \normalsize \texttt{.rodata}: Donde declarar constantes globales inicializadas. \small (\texttt{DB}, \texttt{DW}, \texttt{DD} y \texttt{DQ}).
    \pause
    \item[-] \normalsize \texttt{.bss}: Donde declarar variables globales no inicializadas. \small (\texttt{RESB}, \texttt{RESW}, \texttt{RESD} y \texttt{RESQ}).
    \pause
    \item[-] \normalsize \texttt{.text}: Es donde se escribe el código.
    \end{itemize}
    \bigskip
    \pause
    Etiquetas y símbolos
    \pause
    \begin{itemize}
        \item[-] \texttt{global}: Modificador que define un símbolo que va a ser visto externamente.
        \pause
        \item[-] \texttt{\_start}: Símbolo utilizando como punto de entrada de un programa.
    \end{itemize}
\end{frame}

\begin{frame}[fragile]{Pseudoinstrucciones}
    Comandos e instrucciones para el ensamblador
    \vspace{.4cm}
    \begin{itemize}
    \setlength\itemsep{1em}
    \pause
    \item[-] \texttt{DB}, \texttt{DW}, \texttt{DD}, \texttt{DQ}, \texttt{RESB}, \texttt{RESW}, \texttt{RESD} y \texttt{RESQ}.
    \pause
    \item[-] expresión \verb|$|, se evalúa en la posición en memoria al principio de la línea que contiene la expresión.
    \pause
    \item[-] comando \verb|EQU|, para definir constantes que después no quedan en el archivo objeto.
    \pause
    \item[-] comando \verb|INCBIN|, incluye un binario en un archivo assembler.
    \pause
    \item[-] prefijo \verb|TIMES|, repite una cantidad de veces la instrucción que le sigue.
    \end{itemize}
\end{frame}

\begin{frame}[fragile]{Llamadas al sistema operativo (syscalls)}
    \small
    Utilizando la famosa \texttt{int 0x80} (en Linux) solicitamos al Sistema Operativo que haga algo por nosotros.\\
    \vspace{0.2cm}
    \pause
    Su interfaz es:
    \begin{enumerate}
    \item[1-] El número de función que queremos en rax
    \item[2-] Los parámetros en rbx, rcx, rdx, rsi, rdi y rbp; en ese orden
    \item[3-] Llamamos a la interrupción del sistema operativo (int 0x80)
    \item[4-] En general, la respuesta está en rax
    \end{enumerate}
    \pause
    \begin{itemize}
    \item[-] \textcolor{verdeuca}{Mostrar por pantalla} (\textbf{sys\_write}):\\
    \begin{tabular}{l}
    Función \textbf{4}\\
    Parámetro 1: \textbf{¿donde?} (1 = stdout)\\
    Parámetro 2: \textbf{Dirección de memoria del mensaje}\\
    Parámetro 3: \textbf{Longitud del mensaje} (en bytes)\\
    \end{tabular}
    \pause
    \item[-] \textcolor{verdeuca}{Terminar programa} (\textbf{exit}):\\
    \begin{tabular}{l}
    Función \textbf{1}\\
    Parámetro 1: \textbf{código de retorno} (0 = sin error)\\
    \end{tabular}
    \end{itemize}
\end{frame}

\newcommand{\A}[0]{\begin{tikzpicture} \draw[white] (0,0) rectangle (.4,.4); \draw[white] (0,0) rectangle (.3,.3);\end{tikzpicture}}
\newcommand{\T}[0]{\begin{tikzpicture} \draw[white] (0,0) rectangle (.4,.4); \draw[red,fill=red] (0,0) rectangle (.3,.3);\end{tikzpicture}}
\newcommand{\R}[0]{\begin{tikzpicture} \draw[white] (0,0) rectangle (.4,.4); \draw[verdeuca,fill=verdeuca] (0,0) rectangle (.3,.3);\end{tikzpicture}}
\newcommand{\B}[0]{\begin{tikzpicture} \draw[white] (0,0) rectangle (.4,.4); \draw[naranjauca,fill=naranjauca] (0,0) rectangle (.3,.3);\end{tikzpicture}}

\begin{frame}[fragile]{Hola Mundo... solución}
    \verb|     section .data                  |\\ 
    \verb|       msg: DB 'Hola Mundo', 10     | \A \\
    \verb|       largo EQU $ - msg            |\\
    \verb|                                    |\\
    \verb|       global _start                |\\
    \verb|     section .text                  |\\
    \verb|       _start:                      |\\
    \verb|         mov rax, 4     ; funcion 4 | \A \\
    \verb|         mov rbx, 1     ; stdout    | \A \\
    \verb|         mov rcx, msg   ; mensaje   | \A \\
    \verb|         mov rdx, largo ; longitud  | \A \\
    \verb|         int 0x80                   | \A \\
    \verb|         mov rax, 1     ; funcion 1 | \A \\
    \verb|         mov rbx, 0     ; codigo    | \A \\
    \verb|         int 0x80                   | \A \\
\end{frame}                          

\begin{frame}[fragile]{Hola Mundo... solución}
    \verb|     section .data                  |\\ 
    \verb|       msg: DB 'Hola Mundo', 10     | \A \A \B \B \B \B \B \B \B \B \B \B \T \\
    \verb|       largo EQU $ - msg            |\\
    \verb|                                    |\\
    \verb|       global _start                |\\
    \verb|     section .text                  |\\
    \verb|       _start:                      |\\
    \verb|         mov rax, 4     ; funcion 4 | \A \A  \R \R \B \B \B \B \B \B \B \B \\
    \verb|         mov rbx, 1     ; stdout    | \A \A  \R \R \B \B \B \B \B \B \B \B \\
    \verb|         mov rcx, msg   ; mensaje   | \A \A  \R \R \B \B \B \B \B \B \B \B \\
    \verb|         mov rdx, largo ; longitud  | \A \A  \R \R \B \B \B \B \B \B \B \B \\
    \verb|         int 0x80                   | \A \A  \R \B \\
    \verb|         mov rax, 1     ; funcion 1 | \A \A  \R \R \B \B \B \B \B \B \B \B \\
    \verb|         mov rbx, 0     ; codigo    | \A \A  \R \R \B \B \B \B \B \B \B \B \\
    \verb|         int 0x80                   | \A \A  \R \B \\
\end{frame}

\begin{frame}[fragile]{Ensamblando y linkeando}
    Ensamblamos:\\
    \vspace{0.2cm}
    \verb|      nasm -f elf64 -g -F DWARF holamundo.asm|\\
    \vspace{1cm}
    Linkeamos:\\
    \vspace{0.2cm}
    \verb|      ld -o holamundo holamundo.o|\\
    \vspace{1cm}
    Ejecutamos:\\
    \vspace{0.2cm}
    \verb|      ./holamundo|\\
\end{frame}

\begin{frame}[fragile,t]{Ejecución}
    \small
    \vspace{0.2cm}
        \verb|$ ./holamundo|\\
        \pause
        \vspace{0.2cm}
        \verb|Hola Mundo|\\
        \pause
    \vspace{0.7cm}
        \textcolor{verdeuca}{¿Dondé quedo el código de error?}\\
        \pause
        \vspace{0.2cm}
        \verb|$ echo "$?"|\\
        \pause
        \vspace{0.2cm}
        \verb|0|\\
        \pause
    \vspace{0.7cm}
        \textcolor{verdeuca}{¿Cómo reconozco un binario?}\\
        \pause
        \vspace{0.2cm}
        \verb|$ file holamundo|\\
        \pause
        \vspace{0.2cm}
        \verb|holamundo: ELF 64-bit LSB executable, x86-64, version 1 (SYSV),|\\
        \verb| statically linked, with debug_info, not stripped|\\
\end{frame}

\begin{frame}[fragile,t]{Ejecución}
    \small
    \textcolor{verdeuca}{Ya sabemos que es un \texttt{elf}. Pero ¿Qué tiene dentro?}\\
    \pause
    \vspace{0.2cm}
    \verb|$ objdump -d holamundo|\\
    \vspace{0.2cm}
    \pause
    \verb|holamundo:     file format elf64-x86-64|\\
    \verb||\\
    \verb|Disassembly of section .text:|\\
    \verb||\\
    \verb|00000000004000b0 <_start>:|\\
    \verb|  4000b0:	b8 04 00 00 00       	mov    $0x4,%eax|\\
    \verb|  4000b5:	bb 01 00 00 00       	mov    $0x1,%ebx|\\
    \verb|  4000ba:	48 b9 d8 00 60 00 00 	movabs $0x6000d8,%rcx|\\
    \verb|  4000c1:	00 00 00 |\\
    \verb|  4000c4:	ba 0b 00 00 00       	mov    $0xb,%edx|\\
    \verb|  4000c9:	cd 80                	int    $0x80|\\
    \verb|  4000cb:	b8 01 00 00 00       	mov    $0x1,%eax|\\
    \verb|  4000d0:	bb 00 00 00 00       	mov    $0x0,%ebx|\\
    \verb|  4000d5:	cd 80                	int    $0x80|\\
\end{frame}

\begin{frame}[fragile,t]{Ejecución}
    \small
    \textcolor{verdeuca}{Y la sección de datos}\\
    \pause
    \vspace{0.2cm}
    \verb|$ objdump -D holamundo|\\
    \vspace{0.2cm}
    \pause
    $\dots$\\
    \vspace{0.2cm}
    \verb|Disassembly of section .data:|\\
    \verb||\\
    \verb|00000000006000d8 <msg>:|\\
    \verb|  6000d8:	48 6f                	rex.W outsl %ds:(%rsi),(%dx)|\\
    \verb|  6000da:	6c                   	insb   (%dx),%es:(%rdi)|\\
    \verb|  6000db:	61                   	(bad)  |\\
    \verb|  6000dc:	20 4d 75             	and    %cl,0x75(%rbp)|\\
    \verb|  6000df:	6e                   	outsb  %ds:(%rsi),(%dx)|\\
    \verb|  6000e0:	64 6f                	outsl  %fs:(%rsi),(%dx)|\\
    \verb|  6000e2:	0a                   	.byte 0xa|\\
    \vspace{0.2cm}
    $\dots$
\end{frame}

\begin{frame}[fragile,t]{Ejecución}
    \small
    \textcolor{verdeuca}{Y si quiero encontrar una cadena de texto}\\
    \pause
    \vspace{0.2cm}
    \verb|$ strings holamundo|\\
    \vspace{0.2cm}
    \pause
    \verb|Hola Mundo|\\
    \verb|holamundo.asm|\\
    \verb|NASM 2.13.02|\\
    \vspace{-0.1cm}
    $\dots$\\
    \vspace{0.1cm}
    \verb|largo|\\
    \vspace{-0.1cm}
    $\dots$\\
    \vspace{0.1cm}
    \verb|.symtab|\\
    \verb|.strtab|\\
    \vspace{-0.1cm}
    $\dots$\\
    \vspace{0.1cm}
    \verb|.text|\\
    \verb|.data|\\
    \verb|.debug_aranges|\\
    \verb|.debug_pubnames|\\
    \verb|.debug_info|\\
    \vspace{-0.1cm}
    $\dots$\\
    \vspace{0.1cm}
\end{frame}

\begin{frame}[fragile,t]{Ejecución}
    \small
    \textcolor{verdeuca}{No confundir con los simbolos dentro de un binario.}\\
    \pause
    \vspace{0.2cm}
    \verb|$ nm holamundo|\\
    \vspace{0.2cm}
    \pause
    \verb|00000000006000e3 D __bss_start|\\
    \verb|00000000006000e3 D _edata|\\
    \verb|00000000006000e8 D _end|\\
    \verb|000000000000000b a largo|\\
    \verb|00000000006000d8 d msg|\\
    \verb|00000000004000b0 T _start|\\
    \pause
    \vspace{0.4cm}
    \textcolor{verdeuca}{¿Y donde se aprende como usar estos bonitos comandos?}\\
    \pause
    \vspace{0.2cm}
    En el manual de referencia.\\
    \pause
    \vspace{0.4cm}
    \textcolor{verdeuca}{¿Cómo accedo al manual?}\\
    \pause
    \vspace{0.2cm}
    Mediante el comando \texttt{man}\\
    \vspace{0.2cm}
    \verb|$ man | \fbox{\textcolor{verdeuca}{comando}}
\end{frame}

\begin{frame}[fragile,t]{Ejecución}
    \small
    \verb|$ man nm|\\
    \vspace{0.2cm}
    \tiny
\begin{verbatim}
NM(1)                                     GNU Development Tools                                    NM(1)

NAME
       nm - list symbols from object files

SYNOPSIS
       nm [-A|-o|--print-file-name] [-a|--debug-syms]
          [-B|--format=bsd] [-C|--demangle[=style]]
          [-D|--dynamic] [-fformat|--format=format]
          [-g|--extern-only] [-h|--help]
          [-l|--line-numbers] [--inlines]
          [-n|-v|--numeric-sort]
          [-P|--portability] [-p|--no-sort]
          [-r|--reverse-sort] [-S|--print-size]
          [-s|--print-armap] [-t radix|--radix=radix]
          [-u|--undefined-only] [-V|--version]
          [-X 32_64] [--defined-only] [--no-demangle]
          [--plugin name] [--size-sort] [--special-syms]
          [--synthetic] [--with-symbol-versions] [--target=bfdname]
          [objfile...]

DESCRIPTION
       GNU nm lists the symbols from object files objfile....  If no object files are listed as
       arguments, nm assumes the file a.out.

       For each symbol, nm shows:

       ·   The symbol value, in the radix selected by options (see below), or hexadecimal by default.

       ·   The symbol type.  At least the following types are used; others are, as well, depending on
           the object file format.  If lowercase, the symbol is usually local; if uppercase, the symbol
           is global (external).  There are however a few lowercase symbols that are shown for special
           global symbols ("u", "v" and "w").

           "A" The symbol's value is absolute, and will not be changed by further linking.

           "B"
           "b" The symbol is in the BSS data section.  This section typically contains zero-initialized
               or uninitialized data, although the exact behavior is system dependent.

           "C" The symbol is common.  Common symbols are uninitialized data.  When linking, multiple
               common symbols may appear with the same name.  If the symbol is defined anywhere, the
               common symbols are treated as undefined references.

\end{verbatim}
\end{frame}

\begin{frame}[plain]
    \begin{center}
    \Huge Debugger
    \end{center}
\end{frame}

\begin{frame}[fragile]{\texttt{GDB}}
    Comandos Básicos\\
    \vspace{0.2cm}
    \small
    \begin{tabular}{ll}
    \verb;r | run;             & Ejecuta el programa hasta el primer break\\
    & \\
    \verb;b | break FILE:LINE; & Breakpoint en la línea\\
    \verb;b | break FUNCTION;  & Breakpoint en la función\\
    \verb;info breakpoints;    & Muestra información sobre los breakpoints\\
    & \\
    \verb;c | continue;        & Continúa con la ejecución\\
    \verb;s | step;            & Siguiente línea (Into)\\
    \verb;n | next;            & Siguiente línea (Over)\\
    \verb;si | stepi;          & Siguiente instrucción asm (Into)\\
    \verb;ni | nexti;          & Siguiente instrucción asm (Over)\\
    & \\
    \verb;x/;\textcolor{red}{\texttt{N}}\textcolor{orange}{\texttt{u}}\textcolor{blue}{\texttt{f}} \verb;ADDR;          & Muestra los datos en memoria\\ % i.e. x/4gx $rsp
    \end{tabular}\\
    % \vspace{0.2cm}
    \scriptsize
    \hspace{.55cm} \textcolor{red}{\texttt{N}} = Cantidad (bytes)\\
    \hspace{.55cm} \textcolor{orange}{\texttt{u}} = Unidad \verb;b|h|w|g;\\
    \hspace{.55cm} \hspace{0.37cm} \verb;b:byte, h:word, w:dword, g:qword;\\
    \hspace{.55cm} \textcolor{blue}{\texttt{f}} = Formato \verb;x|d|u|o|f|a;\\
    \hspace{.55cm} \hspace{0.37cm} \verb;x;:hex, \verb;d;:decimal, \verb;u;:decimal sin signo, \verb;o;:octal, \verb;f;:float, \verb;a;:direcciones, \verb;s;:strings, \verb;i;:inst.\\
\end{frame}

\begin{frame}[fragile,t]{\texttt{GDB - Mostrar memoria}}
    \small
    \verb;x/;\textcolor{red}{\texttt{N}}\textcolor{orange}{\texttt{u}}\textcolor{blue}{\texttt{f}} \verb;ADDR;\\
    \vspace{0.2cm}
    \textcolor{red}{\texttt{N}} = Cantidad (bytes)\\
    \textcolor{orange}{\texttt{u}} = Unidad \verb;b|h|w|g;\\
    \hspace{0.37cm} \verb;b:byte, h:word, w:dword, g:qword;\\
    \textcolor{blue}{\texttt{f}} = Formato \verb;x|d|u|o|f|a;\\
    \hspace{0.37cm} \verb;x;:hex, \verb;d;:decimal, \verb;u;:decimal sin signo, \verb;o;:octal, \verb;f;:float,\\
    \hspace{0.37cm} \verb;a;:direcciones, \verb;s;:strings, \verb;i;:instrucciones.\\
    \vspace{0.4cm}
    \pause
    \textcolor{verdeuca}{Ejemplos}
    \vspace{0.2cm}
    \begin{itemize}
    \item[-] \verb|x/3bx | \fbox{\textcolor{verdeuca}{addr}} \hspace{0.1cm} : Tres bytes en hexadecimal.
    \pause
    \item[-] \verb|x/5wd | \fbox{\textcolor{verdeuca}{addr}} \hspace{0.1cm} : Cinco enteros de 32 bits con signo.
    \pause
    \item[-] \verb|x/1ho | \fbox{\textcolor{verdeuca}{addr}} \hspace{0.1cm} : Un número de 16 bits en representación octal.
    \pause
    \item[-] \verb|x/10gf | \fbox{\textcolor{verdeuca}{addr}} \hspace{0.1cm} : Diez doubles.
    \pause
    \item[-] \verb|x/s | \fbox{\textcolor{verdeuca}{addr}} \hspace{0.1cm} : Una string terminada en cero.
    \end{itemize}
\end{frame}

\begin{frame}[fragile]{GDB}
    Configuración de GDB:\\
    \vspace{0.1cm}
    \begin{verbatim}
        ~/.gdbinit 
    \end{verbatim}
    \vspace{0.4cm}
    Para usar sintaxis intel y guardar historial de comandos:\\
    \vspace{0.1cm}
    \begin{verbatim}
        set disassembly-flavor intel
        set history save
    \end{verbatim}
    \vspace{0.4cm}
    Correr \texttt{GDB} con argumentos:
    \vspace{0.1cm}
    \begin{verbatim}
        gdb --args <ejecutable> <arg1> <arg2> ...
    \end{verbatim}
\end{frame}

\begin{frame}[fragile,t]{GDB}
\small
\begin{verbatim}
$ gdb holamundo 
\end{verbatim}
\pause
\begin{verbatim}
GNU gdb (Ubuntu 8.1-0ubuntu3.2) 8.1.0.20180409-git
\end{verbatim}
\scriptsize
\vspace{-0.8cm}
\begin{verbatim}
Copyright (C) 2018 Free Software Foundation, Inc.
License GPLv3+: GNU GPL version 3 or later <http://gnu.org/licenses/gpl.html>
This is free software: you are free to change and redistribute it.
There is NO WARRANTY, to the extent permitted by law.  Type "show copying"
and "show warranty" for details.
This GDB was configured as "x86_64-linux-gnu".
Type "show configuration" for configuration details.
For bug reporting instructions, please see:
<http://www.gnu.org/software/gdb/bugs/>.
Find the GDB manual and other documentation resources online at:
<http://www.gnu.org/software/gdb/documentation/>.
\end{verbatim}
\small
\vspace{-0.8cm}
\begin{verbatim}
For help, type "help".
Type "apropos word" to search for commands related to "word"...
Reading symbols from holamundo...done.
(gdb) 
\end{verbatim}
\end{frame}

\begin{frame}[fragile,t]{GDB}
    \hspace{3cm}
    \begin{minipage}{10cm}
    \vspace{-0.5cm}
    \scriptsize
    \verb|(gdb) | \pause \verb|b _start|\\ \pause
    \verb|Breakpoint 1 at 0x4000b0: file holamundo.asm, line 8.|\\
    \verb|(gdb) | \pause \verb|r|\\ \pause
    \verb|Starting program: /holamundo |\\
    \verb|Breakpoint 1, _start () at holamundo.asm:8   |\\
    \verb|8	    mov rax, 4     ; funcion 4             |\\
    \verb|(gdb) | \pause \verb|n|\\ \pause
    \verb|9	    mov rbx, 1     ; stdout                |\\
    \verb|(gdb) | \pause \\ 
    \verb|10	    mov rcx, msg   ; mensaje           |\\
    \verb|(gdb) | \pause \\ 
    \verb|11	    mov rdx, largo ; longitud          |\\
    \verb|(gdb) | \pause \\ 
    \verb|12	    int 0x80                           |\\
    \verb|(gdb) | \pause \\ 
    \verb|Hola Mundo                                   |\\
    \verb|13	    mov rax, 1     ; funcion 1         |\\
    \verb|(gdb) | \pause \\ 
    \verb|14	    mov rbx, 0     ; codigo            |\\
    \verb|(gdb) | \pause \\ 
    \verb|15	    int 0x80                           |\\
    \verb|(gdb) | \pause \\ 
    \verb|[Inferior 1 (process 20626) exited normally] |\\
    \verb|(gdb)                                        |\\
    \end{minipage}
\end{frame}

\begin{frame}[fragile,t]{GDB}
    \hspace{3cm}
    \begin{minipage}{10cm}
    \vspace{-0.5cm}
    \scriptsize
    \verb|(gdb) | \pause \verb|r|\\
    \verb|Starting program: /holamundo |\\
    \verb|Breakpoint 1, _start () at holamundo.asm:8   |\\
    \verb|8	    mov rax, 4     ; funcion 4             |\\
    \verb|(gdb) | \verb|n|\\ 
    \verb|9	    mov rbx, 1     ; stdout                |\\
    \verb|(gdb) | \\ 
    \verb|10	    mov rcx, msg   ; mensaje           |\\
    \verb|(gdb) |\\ 
    \verb|11	    mov rdx, largo ; longitud          |\\
    \verb|(gdb) |\\ 
    \verb|12	    int 0x80                           |\\
    \verb|(gdb) | \pause \verb|x/s $rcx|\\ \pause
    \verb|0x6000d8 <msg>:	"Hola Mundo\n,"|\\
    \verb|(gdb) | \pause \verb|info reg|\\ \pause
    \verb|rax            0x4	4|\\
    \verb|rbx            0x1	1|\\
    \verb|rcx            0x6000d8	6291672|\\
    \verb|rdx            0xb	11|\\
    \vspace{-0.1cm}
    $\dots$\\
    \vspace{0.1cm}
    \verb|rsp            0x7fffffffdbf0	0x7fffffffdbf0|\\
    \vspace{-0.1cm}
    $\dots$\\
    \vspace{0.1cm}
    \verb|rip            0x4000c9	0x4000c9 <_start+25>|\\
    \verb|eflags         0x202	[ IF ]|\\
    \vspace{-0.1cm}
    $\dots$\\
    \vspace{0.1cm}
    \end{minipage}
\end{frame}

\begin{frame}[fragile,t]{GDB}
    \hspace{3cm}
    \begin{minipage}{10cm}
    \vspace{-0.5cm}
    \scriptsize
    \verb|Breakpoint 1, _start () at holamundo.asm:8|\\
    \verb|8	    mov rax, 4     ; funcion 4 |\\
    \vspace{0.1cm}
    \verb|(gdb) | \pause \verb|list|\\
    \pause
    \vspace{0.1cm}
    \verb|3   largo EQU $ - msg            |\\
    \verb|4                                |\\
    \verb|5   global _start                |\\
    \verb|6 section .text                  |\\
    \verb|7   _start:                      |\\
    \verb|8     mov rax, 4     ; funcion 4 |\\
    \verb|9     mov rbx, 1     ; stdout    |\\
    \verb|10    mov rcx, msg   ; mensaje  |\\ 
    \verb|11    mov rdx, largo ; longitud  |\\
    \verb|12    int 0x80                   |\\
    \vspace{0.1cm}
    \verb|(gdb) | \pause \verb|list|\\
    \pause
    \vspace{0.1cm}
    \verb|13    mov rax, 1     ; funcion 1 |\\
    \verb|14    mov rbx, 0     ; codigo    |\\
    \verb|15    int 0x80                   |\\
    \verb|(gdb) |\\
    \end{minipage}
\end{frame}

\begin{frame}[fragile,t]{GDB}
    \vspace{1cm}
    \scriptsize
    \verb|Breakpoint 1, _start () at holamundo.asm:8|\\
    \verb|8	    mov rax, 4     ; funcion 4 |\\
    \vspace{0.1cm}
    \verb|(gdb) | \pause \verb|x/12i $rip|\\
    \pause
    \vspace{0.1cm}
    \verb|=> 0x4000b0 <_start>:     mov    $0x4,%eax|\\
    \verb|   0x4000b5 <_start+5>:   mov    $0x1,%ebx|\\
    \verb|   0x4000ba <_start+10>:  movabs $0x6000d8,%rcx|\\
    \verb|   0x4000c4 <_start+20>:  mov    $0xb,%edx|\\
    \verb|   0x4000c9 <_start+25>:  int    $0x80|\\
    \verb|   0x4000cb <_start+27>:  mov    $0x1,%eax|\\
    \verb|   0x4000d0 <_start+32>:  mov    $0x0,%ebx|\\
    \verb|   0x4000d5 <_start+37>:  int    $0x80|\\
    \verb|   0x4000d7:	add    %cl,0x6f(%rax)|\\
    \verb|   0x4000da:	insb   (%dx),%es:(%rdi)|\\
    \verb|   0x4000db:	(bad)  |\\
    \verb|   0x4000dc:	and    %cl,0x75(%rbp)|\\
    \vspace{0.1cm}
    \verb|(gdb) |\\
    \begin{textblock}{50}(90,2)
    \begin{block}{\small\texttt{holamundo.asm}}
    \verb|section .data                  |\\ 
    \verb|  msg: DB 'Hola Mundo', 10     |\\
    \verb|  largo EQU $ - msg            |\\
    \verb|                               |\\
    \verb|  global _start                |\\
    \verb|section .text                  |\\
    \verb|  _start:                      |\\
    \verb|    mov rax, 4     ; funcion 4 |\\
    \verb|    mov rbx, 1     ; stdout    |\\
    \verb|    mov rcx, msg   ; mensaje   |\\
    \verb|    mov rdx, largo ; longitud  |\\
    \verb|    int 0x80                   |\\
    \verb|    mov rax, 1     ; funcion 1 |\\
    \verb|    mov rbx, 0     ; codigo    |\\
    \verb|    int 0x80                   |
    \end{block}
    \end{textblock}
\end{frame}

\begin{frame}[plain]
    \begin{center}
    \vspace{2cm}
    \huge ¡Gracias!\\
    \vspace{2cm}
    \normalsize Recuerden leer los comentarios al final de \\ este video por aclaraciones o fe de erratas.
    \end{center}
\end{frame}

\end{document}

