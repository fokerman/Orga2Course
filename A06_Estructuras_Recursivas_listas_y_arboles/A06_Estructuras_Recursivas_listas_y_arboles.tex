% === A06 - Estructuras Recursivas Listas y Arboles ===
% David Alejandro Gonzalez Marquez
% dmarquez@dc.uba.ar / fokerman@gmail.com
% https://github.com/fokerman/Orga2Course

\documentclass[aspectratio=169]{beamer}
% \documentclass[handout]{beamer}

% % % Packages
\usepackage[sfdefault]{AlegreyaSans}
\usepackage{inconsolata}
\usepackage{multicol}
\usepackage{multirow}
\usepackage[spanish]{babel}
\usepackage[utf8]{inputenc}
\usepackage{enumerate}
\usepackage{color}
\usepackage{xcolor}
\usepackage[absolute,overlay]{textpos}
  \setlength{\TPHorizModule}{1mm}
  \setlength{\TPVertModule}{1mm}
\usepackage{framed}
\usepackage{mfirstuc} % para poner en mayusculas la primer letra
\usepackage{xspace} % para crear espacios en comandos 
\usepackage{pbox}
\usepackage{tikz}
\usepackage{mathabx}

% % % Beamer config
\usetheme{Pittsburgh}
\usecolortheme[rgb={1,0.48,0.0}]{structure}
\setbeamercolor{block title}{fg=white,bg=verdeuca}
\xdefinecolor{verdeuca}{rgb}{0.0,0.48,0.54}
\xdefinecolor{naranjauca}{rgb}{1,0.48,0.0}
\setbeamercolor{palette quaternary}{fg=white,bg=verdeuca}
\setbeamertemplate{title page}[default][colsep=-4bp, rounded=true] % remove title shadow
\setbeamertemplate{frametitle}[default][colsep=-2bp, shadow=false] % remove frame title shadow
\setbeamertemplate{navigation symbols}{} % remove navigation symbols
\beamertemplatenavigationsymbolsempty

% % % Colors
\definecolor{AzulClaro}{rgb}{.31,.506,.741}
\definecolor{Gris}{gray}{0.8}
\definecolor{Celeste}{rgb}{.255,.41,.884}
\definecolor{Rojo}{rgb}{1, 0, 0}

% % % Rename
\newcommand{\tab}[0]{\hspace{15pt}}

% % % Blocks
\setbeamercolor{block body}{fg=black, bg=black!10}
\setbeamercolor{block title}{fg=black, bg=black!20}
\setbeamercolor{coloredboxstuffNaranja}{fg=naranjauca,bg=black!10} %% PARA LOS BOX
\setbeamercolor{coloredboxstuffVerde}{fg=verdeuca,bg=black!10} %% PARA LOS BOX

% % % Start

\title{\Huge Estructuras Recursivas}
\subtitle{Listas y Árboles}
\author{David Alejandro González Márquez}
\institute{Departamento de Computación\\
Facultad de Ciencias Exactas y Naturales\\
Universidad de Buenos Aires}
\date{}

\begin{document}

\begin{frame}[plain]
    \titlepage 
\end{frame}

\begin{frame}[fragile]
    \frametitle{Nota: \texttt{typedef}}
    \large
    \begin{itemize}
    \setlength\itemsep{0.8cm}
    \item<1->[-] En C es posible definir nuevos tipos de datos mediante el uso de \textcolor{naranjauca}{\texttt{typedef}}.
    \item<2->[-] Ejemplo:\\
    \hspace{2cm}\textcolor{naranjauca}{\texttt{typedef }}\textcolor{verdeuca}{\texttt{int}}\textcolor{naranjauca}{\texttt{ numero\_t}}
    \item<3->[-] Podría ahora declarar una función como:\\
    \vspace{0.5cm}
    \hspace{2cm}\textcolor{naranjauca}{\texttt{numero\_t}}\textcolor{verdeuca}{\texttt{ suma(}}\textcolor{naranjauca}{\texttt{numero\_t}}\textcolor{verdeuca}{\texttt{ a,}}\textcolor{naranjauca}{\texttt{ numero\_t}}\textcolor{verdeuca}{\texttt{ b)}}
    \item<4->[-] Para el ejemplo, \textcolor{naranjauca}{\texttt{numero\_t}} es un nuevo tipo, sinónimo de \textcolor{verdeuca}{\texttt{int}}.
    \end{itemize}
\end{frame}

\begin{frame}[fragile]
    \frametitle{Nota: \texttt{typedef} en \texttt{struct}'s}
    \large
    \begin{itemize}
    \setlength\itemsep{0.3cm}
    \item<1->[-] Podemos usar \textcolor{naranjauca}{\texttt{typedef}} para renombrar \textcolor{naranjauca}{\texttt{struct}'s}
    \item<2->[-] Ejemplo:\\
    \vspace{-0.5cm}
    \normalsize
    \hspace{4cm}\textcolor{verdeuca}{\texttt{struct alumno \{}}\\
    \hspace{5cm}\textcolor{verdeuca}{\texttt{char* nombre;}}\\
    \hspace{5cm}\textcolor{verdeuca}{\texttt{char comision;}}\\
    \hspace{5cm}\textcolor{verdeuca}{\texttt{int dni;}}\\
    \hspace{4cm}\textcolor{verdeuca}{\texttt{\};}}
    \item<3->[-] \large Se escribiría como:\\
    \vspace{-0.5cm}
    \normalsize
    \hspace{4cm}\textcolor{naranjauca}{\texttt{typedef }}\textcolor{verdeuca}{\texttt{struct \{}}\\
    \hspace{5cm}\textcolor{verdeuca}{\texttt{char* nombre;}}\\
    \hspace{5cm}\textcolor{verdeuca}{\texttt{char comision;}}\\
    \hspace{5cm}\textcolor{verdeuca}{\texttt{int dni;}}\\
    \hspace{4cm}\textcolor{verdeuca}{\texttt{\}}}\textcolor{naranjauca}{\texttt{ alumno\_t}}\textcolor{verdeuca}{\texttt{;}}
    \item<4->[-] \large \textcolor{naranjauca}{\texttt{ alumno\_t}} es un nuevo tipo y puede usarse en remplazo de \textcolor{verdeuca}{\texttt{struct alumno}}
    \end{itemize}
\end{frame}

\begin{frame}[fragile]
    \frametitle{Nota: \texttt{typedef} en tipos de funciones}
    \large
    \begin{itemize}
    \setlength\itemsep{0.4cm}
    \item<1->[-] En C las funciones en sí mismas tiene un tipo de datos.
    \item<2->[-] Vamos a utilizar \textcolor{naranjauca}{\texttt{typedef}} para nombrar al tipo de una función.
    \item<3->[-] Ejemplo:\\
    \normalsize
    \hspace{4cm}\textcolor{verdeuca}{\texttt{int suma(int, int);}}\\
    \item<4->[-] \large Se escribiría como:\\
    \hspace{4cm}\textcolor{naranjauca}{\texttt{typedef }}\textcolor{verdeuca}{\texttt{int (*func\_suma)(int, int);}}\\
    \vspace{0.4cm}
    \item<5->[-] Podemos utilizar \textcolor{verdeuca}{\texttt{func\_suma}} como el tipo de datos del puntero a una función que toma dos enteros y retorna un entero.
    \end{itemize}
\end{frame}

\begin{frame}[t]
    \frametitle{Listas}
    \begin{textblock}{500}(5,15) \only<1>{Lista simplemente enlazada:} \end{textblock}
    \begin{textblock}{500}(5,15) \only<2>{Lista doblemente enlazada:} \end{textblock}
    \begin{textblock}{500}(5,15) \only<3>{Lista doblemente enlazada (representación simplificada):} \end{textblock}
    \begin{textblock}{500}(5,25) \only<1>{\includegraphics[scale=0.7]{img/listas-layer1.pdf}} \end{textblock}
    \begin{textblock}{500}(5,25) \only<2>{\includegraphics[scale=0.7]{img/listas-layer2.pdf}} \end{textblock}
    \begin{textblock}{500}(5,25) \only<3>{\includegraphics[scale=0.7]{img/listas-layer3.pdf}} \end{textblock}
    \begin{textblock}{500}(5,55) \only<1>{
    \hspace{1cm}\textcolor{verdeuca}{\texttt{typedef struct \{}}\\
    \hspace{2cm}\textcolor{verdeuca}{\texttt{int dato;}}\\
    \hspace{2cm}\textcolor{verdeuca}{\texttt{node\_t* proximo;}}\\
    \hspace{1cm}\textcolor{verdeuca}{\texttt{\} node\_t;}}
    } \end{textblock}
    \begin{textblock}{500}(5,55) \only<2-3>{
    \hspace{1cm}\textcolor{verdeuca}{\texttt{typedef struct \{}}\\
    \hspace{2cm}\textcolor{verdeuca}{\texttt{int dato;}}\\
    \hspace{2cm}\textcolor{verdeuca}{\texttt{node\_t* proximo;}}\\
    \hspace{2cm}\textcolor{verdeuca}{\texttt{node\_t* anterior;}}\\
    \hspace{1cm}\textcolor{verdeuca}{\texttt{\} node\_t;}}
    } \end{textblock}
\end{frame}

\begin{frame}[t]
    \frametitle{Listas}
    \begin{textblock}{500}(5,15) \only<1>{Lista circular simplemente enlazada:} \end{textblock}
    \begin{textblock}{500}(5,15) \only<2>{Lista circular doblemente enlazada:} \end{textblock}
    \begin{textblock}{500}(5,25) \only<1>{\includegraphics[scale=0.7]{img/listacircular-layer1.pdf}} \end{textblock}
    \begin{textblock}{500}(5,25) \only<2>{\includegraphics[scale=0.7]{img/listacircular-layer2.pdf}} \end{textblock}
    \begin{textblock}{500}(5,55) \only<1>{
    \hspace{1cm}\textcolor{verdeuca}{\texttt{typedef struct \{}}\\
    \hspace{2cm}\textcolor{verdeuca}{\texttt{int dato;}}\\
    \hspace{2cm}\textcolor{verdeuca}{\texttt{node\_t* proximo;}}\\
    \hspace{1cm}\textcolor{verdeuca}{\texttt{\} node\_t;}}
    } \end{textblock}
    \begin{textblock}{500}(5,55) \only<2>{
    \hspace{1cm}\textcolor{verdeuca}{\texttt{typedef struct \{}}\\
    \hspace{2cm}\textcolor{verdeuca}{\texttt{int dato;}}\\
    \hspace{2cm}\textcolor{verdeuca}{\texttt{node\_t* proximo;}}\\
    \hspace{2cm}\textcolor{verdeuca}{\texttt{node\_t* anterior;}}\\
    \hspace{1cm}\textcolor{verdeuca}{\texttt{\} node\_t;}}
    } \end{textblock}
\end{frame}

\begin{frame}
    \frametitle{Listas}
    \begin{textblock}{500}(5,15) \only<1>{Lista con puntero al primer elemento:} \end{textblock}
    \begin{textblock}{500}(5,15) \only<2>{Lista con puntero al primer y último elemento:} \end{textblock}
    \begin{textblock}{500}(5,15) \only<3>{Lista con puntero al primer elemento, último elemento y tamaño:} \end{textblock}
    \begin{textblock}{50}(15,22) \only<1>{\includegraphics[scale=0.7]{img/listastipo-layer1.pdf}} \end{textblock}
    \begin{textblock}{50}(15,22) \only<2>{\includegraphics[scale=0.7]{img/listastipo-layer2.pdf}} \end{textblock}
    \begin{textblock}{50}(15,22) \only<3>{\includegraphics[scale=0.7]{img/listastipo-layer3.pdf}} \end{textblock}
    \begin{textblock}{500}(15,60) \only<1>{
    \hspace{1cm}\textcolor{verdeuca}{\texttt{typedef struct \{}}\\
    \hspace{2cm}\textcolor{verdeuca}{\texttt{node\_t* primero;}}\\
    \hspace{1cm}\textcolor{verdeuca}{\texttt{\} list\_t;}}
    } \end{textblock}
    \begin{textblock}{500}(15,60) \only<2>{
    \hspace{1cm}\textcolor{verdeuca}{\texttt{typedef struct \{}}\\
    \hspace{2cm}\textcolor{verdeuca}{\texttt{node\_t* primero;}}\\
    \hspace{2cm}\textcolor{verdeuca}{\texttt{node\_t* ultimo;}}\\
    \hspace{1cm}\textcolor{verdeuca}{\texttt{\} list\_t;}}
    } \end{textblock}
    \begin{textblock}{500}(15,60) \only<3>{
    \hspace{1cm}\textcolor{verdeuca}{\texttt{typedef struct \{}}\\
    \hspace{2cm}\textcolor{verdeuca}{\texttt{node\_t* primero;}}\\
    \hspace{2cm}\textcolor{verdeuca}{\texttt{node\_t* ultimo;}}\\
    \hspace{2cm}\textcolor{verdeuca}{\texttt{int cantidad;}}\\
    \hspace{1cm}\textcolor{verdeuca}{\texttt{\} list\_t;}}
    } \end{textblock}
    \begin{textblock}{500}(75,60) \only<1->{
    \hspace{1cm}\textcolor{verdeuca}{\texttt{typedef struct \{}}\\
    \hspace{2cm}\textcolor{verdeuca}{\texttt{int dato;}}\\
    \hspace{2cm}\textcolor{verdeuca}{\texttt{node\_t* proximo;}}\\
    \hspace{2cm}\textcolor{verdeuca}{\texttt{node\_t* anterior;}}\\
    \hspace{1cm}\textcolor{verdeuca}{\texttt{\} node\_t;}}
    } \end{textblock}
\end{frame}

\begin{frame}
    \frametitle{Datos}
    \begin{textblock}{500}(15,11) \only<1-3>{\includegraphics[scale=0.7]{img/listasdato-layer1.pdf}} \end{textblock}
    \begin{textblock}{500}(15,38) \only<2-3>{\includegraphics[scale=0.7]{img/listasdato-layer2.pdf}} \end{textblock}
    \begin{textblock}{500}(15,65) \only<3-3>{\includegraphics[scale=0.7]{img/listasdato-layer4.pdf}} \end{textblock}
    \begin{textblock}{500}(86,10) \only<1-3>{Un \texttt{int} en el nodo.\\
    \small
    \hspace{1cm}\textcolor{verdeuca}{\texttt{typedef struct \{}}\\
    \hspace{2cm}\textcolor{naranjauca}{\texttt{int dato;}}\\
    \hspace{2cm}\textcolor{verdeuca}{\texttt{node\_t* proximo;}}\\
    \hspace{1cm}\textcolor{verdeuca}{\texttt{\} node\_t;}}
    } \end{textblock}
    \begin{textblock}{500}(86,37) \only<2-3>{Un puntero a un \texttt{int}.\\
    \small
    \hspace{1cm}\textcolor{verdeuca}{\texttt{typedef struct \{}}\\
    \hspace{2cm}\textcolor{naranjauca}{\texttt{int* dato;}}\\
    \hspace{2cm}\textcolor{verdeuca}{\texttt{node\_t* proximo;}}\\
    \hspace{1cm}\textcolor{verdeuca}{\texttt{\} node\_t;}}
    } \end{textblock}
    \begin{textblock}{500}(86,64) \only<3-3>{Un puntero a una estructura.\\
    \small
    \hspace{1cm}\textcolor{verdeuca}{\texttt{typedef struct \{}}\\
    \hspace{2cm}\textcolor{naranjauca}{\texttt{alumno\_t* dato;}}\\
    \hspace{2cm}\textcolor{verdeuca}{\texttt{node\_t* proximo;}}\\
    \hspace{1cm}\textcolor{verdeuca}{\texttt{\} node\_t;}}
    } \end{textblock}
    \begin{textblock}{500}(15,10) \only<4->{\includegraphics[scale=0.7]{img/listasdato-layer3.pdf}} \end{textblock}
    \begin{textblock}{500}(15,35) \only<5->{\includegraphics[scale=0.7]{img/listasdato-layer5.pdf}} \end{textblock}
    \begin{textblock}{500}(15,57) \only<6->{\includegraphics[scale=0.7]{img/listasdato-layer6.pdf}} \end{textblock}
    \begin{textblock}{500}(86,12) \only<4->{Una estructura como parte del nodo.\\
    \small
    \hspace{1cm}\textcolor{verdeuca}{\texttt{typedef struct \{}}\\
    \hspace{2cm}\textcolor{naranjauca}{\texttt{alumno\_t dato;}}\\
    \hspace{2cm}\textcolor{verdeuca}{\texttt{node\_t* proximo;}}\\
    \hspace{1cm}\textcolor{verdeuca}{\texttt{\} node\_t;}}
    } \end{textblock}
    \begin{textblock}{500}(86,37) \only<5->{Un puntero a \texttt{void} (sin tipo).\\
    \small
    \hspace{1cm}\textcolor{verdeuca}{\texttt{typedef struct \{}}\\
    \hspace{2cm}\textcolor{naranjauca}{\texttt{void* dato;}}\\
    \hspace{2cm}\textcolor{verdeuca}{\texttt{node\_t* proximo;}}\\
    \hspace{1cm}\textcolor{verdeuca}{\texttt{\} node\_t;}}
    } \end{textblock}
    \begin{textblock}{500}(86,59) \only<6->{Un puntero a \texttt{void} y un puntero a una función.\\
    \small
    \hspace{1cm}\textcolor{verdeuca}{\texttt{typedef struct \{}}\\
    \hspace{2cm}\textcolor{naranjauca}{\texttt{void* dato;}}\\
    \hspace{2cm}\textcolor{naranjauca}{\texttt{func\_t* funcion;}}\\
    \hspace{2cm}\textcolor{verdeuca}{\texttt{node\_t* proximo}}\\
    \hspace{1cm}\textcolor{verdeuca}{\texttt{\} node\_t;}}
    } \end{textblock}
\end{frame}

\begin{frame}
    \frametitle{Árboles}
    \begin{textblock}{500}(15,15) \only<1-4>{Árbol binario:} \end{textblock}
    \begin{textblock}{500}(15,15) \only<5>{Árbol binario (representación simplificada):} \end{textblock}
    \begin{textblock}{500}(15,15) \only<6>{Árbol binario con puntero al padre:} \end{textblock}
    \begin{textblock}{500}(15,15) \only<7>{Representación de un árbol binario:} \end{textblock}
    \begin{textblock}{125}(15,22) \uncover<1-3>{
    \hspace{3cm}\textcolor{verdeuca}{\texttt{typedef struct \{}}\\
    \hspace{4cm}\textcolor{verdeuca}{\texttt{int dato}}\\
    \hspace{4cm}\textcolor{verdeuca}{\texttt{node\_t* derecha}}\\
    \hspace{4cm}\textcolor{verdeuca}{\texttt{node\_t* izquierda}}\\
    \hspace{3cm}\textcolor{verdeuca}{\texttt{\} node\_t;}}\\
    \vspace{0.4cm}
    }
    \begin{itemize}
    \setlength\itemsep{0.4cm}
     \item<1-3>[-] Cada nodo define un par de punteros, uno a derecha y otro a izquierda.
     \item<2-3>[-] \textbf{Árbol binario de busqueda}: Todos los datos a derecha son más grandes que el dato en la raíz y todos los datos a izquierda son menores o iguales al de la raíz.
     \item<3-3>[-] \textbf{Balanceado}: Para todos los nodos, la cantidad de nodos de cada lado del arbol es equivalente.
    \end{itemize}
    \end{textblock}
    \begin{textblock}{50}(10,24) \only<4>{\includegraphics[scale=0.6]{img/arbol-layer1.pdf}} \end{textblock}
    \begin{textblock}{50}(10,24) \only<5>{\includegraphics[scale=0.6]{img/arbol-layer2.pdf}} \end{textblock}
    \begin{textblock}{50}(10,24) \only<6>{\includegraphics[scale=0.6]{img/arbol-layer3.pdf}} \end{textblock}
    \begin{textblock}{50}(10,24) \only<7>{\includegraphics[scale=0.6]{img/arbol-layer4.pdf}} \end{textblock}
\end{frame}

\begin{frame}
    \frametitle{Árboles n-arios}
    \begin{textblock}{500}(15,12) \only<1-10>{Árbol n-ario:} \end{textblock}
    \begin{textblock}{500}(15,12) \only<11-12>{Árbol n-ario (representación simplificada):} \end{textblock}
    \begin{textblock}{500}(15,20) \only<1>{
    \hspace{1cm}\textcolor{verdeuca}{\texttt{typedef struct \{}}\\
    \hspace{2cm}\textcolor{verdeuca}{\texttt{int dato}}\\
    \hspace{2cm}\textcolor{verdeuca}{\texttt{node\_t* proximo}}\\
    \hspace{2cm}\textcolor{verdeuca}{\texttt{node\_t* hijo}}\\
    \hspace{1cm}\textcolor{verdeuca}{\texttt{\} node\_t;}}\\
    \vspace{0.4cm}
    \begin{itemize}
    \setlength\itemsep{0.4cm}
     \item[-] Cada nivel funciona como una lista.
     \item[-] El puntero hijos indica el nivel inferior.
     \item[-] Nivel puede tener una cantidad arbitraria de nodos.
    \end{itemize}
    } \end{textblock}
    \begin{textblock}{50}(3,20) \only<10-10>{\includegraphics[scale=0.435]{img/arbolnario-layer1.pdf}} \end{textblock}
    \begin{textblock}{50}(3,20) \only<2-10>{\includegraphics[scale=0.435]{img/arbolnario-layer2.pdf}} \end{textblock}
    \begin{textblock}{50}(3,20) \only<3-10>{\includegraphics[scale=0.435]{img/arbolnario-layer3.pdf}} \end{textblock}
    \begin{textblock}{50}(3,20) \only<4-10>{\includegraphics[scale=0.435]{img/arbolnario-layer4.pdf}} \end{textblock}
    \begin{textblock}{50}(3,20) \only<5-10>{\includegraphics[scale=0.435]{img/arbolnario-layer5.pdf}} \end{textblock}
    \begin{textblock}{50}(3,20) \only<6-10>{\includegraphics[scale=0.435]{img/arbolnario-layer6.pdf}} \end{textblock}
    \begin{textblock}{50}(3,20) \only<7-10>{\includegraphics[scale=0.435]{img/arbolnario-layer7.pdf}} \end{textblock}
    \begin{textblock}{50}(3,20) \only<8-10>{\includegraphics[scale=0.435]{img/arbolnario-layer8.pdf}} \end{textblock}
    \begin{textblock}{50}(3,20) \only<9-10>{\includegraphics[scale=0.435]{img/arbolnario-layer9.pdf}} \end{textblock}
    \begin{textblock}{50}(3,20) \only<11>{\includegraphics[scale=0.435]{img/arbolnario-layer10.pdf}} \end{textblock}
    \begin{textblock}{50}(3,20) \only<12>{\includegraphics[scale=0.435]{img/arbolnario-layer11.pdf}} \end{textblock}
\end{frame}

\begin{frame}[fragile,t]
    \frametitle{Nota sobre doble punteros}
    \begin{textblock}{100}(3,14) \only<1->{\includegraphics[scale=0.6]{img/punterodoble-layer1.pdf}} \end{textblock} % gris 1
    \begin{textblock}{100}(3,14) \only<3->{\includegraphics[scale=0.6]{img/punterodoble-layer2.pdf}} \end{textblock} % gris 2
    \begin{textblock}{100}(3,14) \only<7->{\includegraphics[scale=0.6]{img/punterodoble-layer3.pdf}} \end{textblock} % gris 3
    \begin{textblock}{100}(3,14) \only<2->{\includegraphics[scale=0.6]{img/punterodoble-layer4.pdf}} \end{textblock} % alu
    \begin{textblock}{100}(3,14) \only<4->{\includegraphics[scale=0.6]{img/punterodoble-layer5.pdf}} \end{textblock} % p_alu
    \begin{textblock}{100}(3,14) \only<8->{\includegraphics[scale=0.6]{img/punterodoble-layer6.pdf}} \end{textblock} % pp_alu
    \begin{textblock}{100}(3,14) \only<1->{\includegraphics[scale=0.6]{img/punterodoble-layer7.pdf}} \end{textblock} % div 1
    \begin{textblock}{100}(3,14) \only<3->{\includegraphics[scale=0.6]{img/punterodoble-layer8.pdf}} \end{textblock} % div 2
    \begin{textblock}{100}(3,14) \only<7->{\includegraphics[scale=0.6]{img/punterodoble-layer9.pdf}} \end{textblock} % div 3
    \begin{textblock}{100}(3,14) \only<6->{\includegraphics[scale=0.6]{img/punterodoble-layer10.pdf}} \end{textblock} % flecha p_alu
    \begin{textblock}{100}(3,14) \only<10->{\includegraphics[scale=0.6]{img/punterodoble-layer11.pdf}} \end{textblock} % flecha pp_alu
    \begin{textblock}{100}(3,14) \only<5->{\includegraphics[scale=0.6]{img/punterodoble-layer12.pdf}} \end{textblock} % puntero p_alu
    \begin{textblock}{100}(3,14) \only<9->{\includegraphics[scale=0.6]{img/punterodoble-layer13.pdf}} \end{textblock} % puntero pp_alu
    \begin{textblock}{100}(57,14)
    \begin{itemize} 
    \setlength\itemsep{0.1cm}
    \small \item<2->[En C  ] \small \only<2->{ \textcolor{gray}{ \texttt{alumno\_t alu;}}}
    \small \item<2->[En ASM] \small \only<2->{ \texttt{push rbp} }
    \small \item<2->[      ] \small \only<2->{ \texttt{mov rbp, rsp} }
    \small \item<2->[      ] \small \only<2-3>{ \texttt{sub rsp, 16;} }\scriptsize \only<2-3>{\texttt{ variable local en la pila (alumno\_t)}} 
                             \small \only<4->{ \texttt{sub rsp, 24;}}\scriptsize \only<4->{\texttt{ variable local en la pila (alumno\_t y alumno\_t*)}}
    \vspace{0.4cm}
    \small \item<4->[En C  ] \small \only<4->{ \textcolor{gray}{ \texttt{alumno\_t *p\_alu}}}\only<1->{ \textcolor{gray}{ \texttt{ = \&alu;}}}
    \small \item<4->[En ASM] \small \only<4->{ \texttt{lea rax, [rbp-16];}}\scriptsize \only<4->{\texttt{ obtener la dirección de alu}}
    \small \item<4->[      ] \small \only<4->{ \texttt{mov [rbp-24], rax;}}\scriptsize \only<4->{\texttt{ escribir la dirección en una variable}}
    \vspace{0.8cm}
    \small \item<8->[En C  ] \small \only<8->{ \textcolor{gray}{ \texttt{alumno\_t **pp\_alu}}}\only<8->{ \textcolor{gray}{ \texttt{ = \&p\_alu;}}}
    \small \item<8->[En ASM] \small \only<8->{ \texttt{lea rdi, [rbp-24];}}\scriptsize \only<8->{\texttt{ obtener la dirección de la variable}}
    \vspace{0.2cm}
    \end{itemize}
    \end{textblock}
\end{frame}

\begin{frame}[fragile]
    \frametitle{Nota sobre como recorrer una lista con un doble puntero}
    \begin{textblock}{100}(3,14) \only<1->{\includegraphics[scale=0.6]{img/listaspunterodoble-layer1.pdf}} \end{textblock} % lista
    \begin{textblock}{100}(3,14) \only<2-7>{\includegraphics[scale=0.6]{img/listaspunterodoble-layer2.pdf}} \end{textblock} % puntero
    \begin{textblock}{100}(3,14) \only<8>{\includegraphics[scale=0.6]{img/listaspunterodoble-layer3.pdf}} \end{textblock} % pila
    \begin{textblock}{100}(3,14) \only<3>{\includegraphics[scale=0.6]{img/listaspunterodoble-layer4.pdf}} \end{textblock} % doble (0)
    \begin{textblock}{100}(3,14) \only<4>{\includegraphics[scale=0.6]{img/listaspunterodoble-layer5.pdf}} \end{textblock} % doble 1
    \begin{textblock}{100}(3,14) \only<5>{\includegraphics[scale=0.6]{img/listaspunterodoble-layer6.pdf}} \end{textblock} % doble 2
    \begin{textblock}{100}(3,14) \only<6>{\includegraphics[scale=0.6]{img/listaspunterodoble-layer7.pdf}} \end{textblock} % doble 3
    \begin{textblock}{100}(3,14) \only<7>{\includegraphics[scale=0.6]{img/listaspunterodoble-layer8.pdf}} \end{textblock} % doble 4
    \begin{textblock}{100}(3,14) \only<3>{\includegraphics[scale=0.6]{img/listaspunterodoble-layer9.pdf}} \end{textblock} % node 1
    \begin{textblock}{100}(3,14) \only<4>{\includegraphics[scale=0.6]{img/listaspunterodoble-layer10.pdf}} \end{textblock} % node 2
    \begin{textblock}{100}(3,14) \only<5>{\includegraphics[scale=0.6]{img/listaspunterodoble-layer11.pdf}} \end{textblock} % node 3
    \begin{textblock}{100}(3,14) \only<6>{\includegraphics[scale=0.6]{img/listaspunterodoble-layer12.pdf}} \end{textblock} % node 4
    \begin{textblock}{100}(3,40)
    \begin{itemize}
    \setlength\itemsep{0.2cm}
    \item<2->[-] \small Suponer un puntero al primer elemento de la lista.
    \item<3->[-] \small Para recorrer la lista, obtenemos dos punteros:
    \begin{itemize}
    \footnotesize
     \item[1.] Doble puntero al primer nodo
     \item[2.] Puntero al primer nodo
    \end{itemize}
    \item<4->[-] \small Luego, iteramos moviendo ambos punteros sobre la lista.
    \item<7->[-] \small El final de la lista será cuando el doble puntero apunte a \texttt{null}
    \item<8->[-] \small Considerar que el doble puntero puede ser una posición en la pila.
    \end{itemize}
    \end{textblock}
\end{frame}

\begin{frame}[fragile]
    \frametitle{Nota sobre estructuras}
    \large
    \begin{itemize}
    \setlength\itemsep{0.4cm}
    \item<1->[-] Las estructuras pueden ser tan complejas como queramos.
    \item<2->[-] Tener en cuenta:
    \begin{itemize}
    \vspace{0.4cm}
    \setlength\itemsep{0.4cm}
     \item<2->[$\cdot$] ¿Qué es parte de la estructura? y ¿Qué es un puntero?
     \item<3->[$\cdot$] ¿Qué tamaño tiene y en que lugar de memoria esta?
     \item<4->[$\cdot$] ¿Tiene punteros a funciones?, ¿Cuáles son?, y ¿Para qué sirven?, ¿Está definidas?
     \item<5->[$\cdot$] Recordar, dibujar las estructuras y los punteros.
    \end{itemize}
    \end{itemize}
\end{frame}

\begin{frame}[fragile]
    \frametitle{Bibliografía: Fuentes y material adicional}
    \begin{itemize}
    \item Convenciones de llamados a función en x86: \\
    \url{https://en.wikipedia.org/wiki/X86_calling_conventions}
    \item Notas sobre System V ABI: \\
    \url{https://wiki.osdev.org/System_V_ABI}
    \item Documentación de NASM: \\
    \url{https://nasm.us/doc/}
    \item Artículo sobre el flag \texttt{-pie}: \\
    \url{https://eklitzke.org/position-independent-executables}
    \item Documentación de System V ABI: \\
    \url{https://uclibc.org/docs/psABI-x86_64.pdf}
    \item Manuales de Intel: \\
    \url{https://software.intel.com/en-us/articles/intel-sdm}
    \end{itemize}
\end{frame}

\begin{frame}[plain]
    \begin{center}
    \vspace{2cm}
    \huge ¡Gracias!\\
    \vspace{2cm}
    \normalsize Recuerden leer los comentarios al final de \\ este video por aclaraciones o fe de erratas.
    \end{center}
\end{frame}

\end{document}
